\chapter{Conclusão}\label{cap:conclusao}
Este trabalho teve como principal objetivo a construção de um projeto de software voltado para o atendimento de crianças com transtorno do espectro autista. Para atingir este objetivo, foi necessário o estudo de temas como: autismo; e-learning; e gamificação.

Sobre o autismo, este trabalho apresentou algumas técnicas de tratamento e atividades as quais ajudam no estímulo de habilidades diversas em crianças diagnosticadas com o transtorno. Outro estudo relevante foi sobre as mecânicas relacionadas à gamificação, proposta por \cite{zichermann2011gamification}, em conjunto das técnicas aplicadas a ambientes de aprendizado via e-learning.

Através do background de conceitos, foi pensado em um software voltado para o auxílio ao tratamento de crianças com TEA utilizando técnicas de e-learning e gamificação. Para produzir os resultados deste trabalho, foi utilizado o processo de engenharia de software da IBM (RUP), através deste, foram fabricados artefatos de software como: diagrama de casos de uso; diagrama de classes; prototipações das telas; e o detalhamento dos requisitos funcionais do sistema.


\section{Trabalhos Futuros}
Como trabalhos futuros poderá ser desenvolvidas aplicações e realizado os testes em pacientes com transtorno do espectro autista, com isso, poderá ser investigado quais as mecânicas de gamificação se adaptam melhor para os tipos de usuários do sistema. 