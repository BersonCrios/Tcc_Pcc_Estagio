% ----------------------------------------------------------
% Introdução (exemplo de capítulo sem numeração, mas presente no Sumário)
% ----------------------------------------------------------

\chapter{Introdução}\label{intro}
A tecnologia da informação está presente em diversas áreas do conhecimento e uma delas é a educação através do \textit{e-learning}, a qual se trata de informações fornecidas por meio de um dispositivo digital como por exemplo: computadores, celulares ou tablets, as quais atuam no apoio do ensino-aprendizado \cite{clark2016learning}. Tendo em vista que cada vez mais os aparelhos de celulares, computadores e até mesmo a rede mundial de computadores, estão em plena expansão, o \textit{e-learning} tem se tornado uma alternativa ao ensino convencional.
%https://navy-training-transformation.wikispaces.com/file/view/e-Learning.pdf/525050486/e-Learning.pdf

Outra área que a tecnologia da informação é bastante presente é a de  jogos, principalmente com a era dos jogos digitais, com videogames e computadores. O jogo trata-se de uma atividade voluntária, realizada em um limite de tempo e em um espaço definido, que são regidas por regras acordadas e obrigatórias com um fim em si mesmo e é acompanhado de tensão, alegria e o fato de saber que está se fazendo algo diferente do dia a dia \cite{huizinga1971homo}. Diretamente relacionado aos jogos, temos o termo gamificação, que trata do uso de mecânicas presentes nos jogos para envolver o usuário a resolver problemas fora do contexto de jogos \cite{zichermann2011gamification}. 
%http://jnsilva.ludicum.org/Huizinga_HomoLudens.pdf

A gamificação vem sendo utilizada em diversas áreas, dentre elas: a empresarial e a educacional. Nas empresas a gamificação é usada para ajudar de forma motivacional, instigando os funcionários em suas metas, e também no relacionamento com os clientes, podendo atuar como meio de divulgar a própria empresa \cite{da2013investigando}. Na educação, a gamificaçāo pode ajudar em diversos fatores: tornar as aulas mais atraentes ao aluno e fazer com que o aluno se torne auto-suficiente a resolver situações problemáticas com o uso da criatividade. Com os desafios e a busca do estudante por premiações ou avanços de níveis, que são algumas das mecânicas presentes na gamificação, ele é motivado a aprender constantemente até dominar o tema por completo, já o erro, se torna também um motivador, pois caso o estudante comece a errar ele passará a ter perdas de pontos. Geralmente os alunos desenvolvem certa facilidade ao ter que passar o conhecimento adquirido a diante \cite{da2014gamificaccao}.
%Gamefication by design

 A gamificação e o e-leaning podem ser utilizadas em conjunto para dar suporte ao tratamento de diversas patologias e transtornos relacionados ao desenvolvimento intelectual, sendo uma delas, o transtorno do espectro autista. Segundo a \citeonline{dadosOnuAutismo}, no mundo hoje, uma em cada 160 crianças nasce com esse transtorno e é apresentado em sua grande maioria em crianças do sexo masculino e, no Brasil, segundo a \citeonline{dadosApaeAutismo}, estima-se que existam mais de dois milhões de autistas.
%http://www.who.int/mediacentre/factsheets/autism-spectrum-disorders/en/dadosApaeAutismo
%https://apaebrasil.org.br/noticia/numero-de-pessoas-com-autismo-aumenta-em-todo-o-brasil em 27/05/18

Pensando nos tratamentos convencionais desenvolvidos por profissionais como: psicólogos, terapeutas ocupacionais e quaisquer outros que trabalhem diretamente com crianças diagnosticadas com transtorno do espectro autista, este trabalho propõe um projeto de uma aplicação que irá usar de mecânicas e de princípios da gamificação e do \textit{e-learning} para auxiliar os profissionais em seus atendimentos e, consequentemente, tornar mais prazerosa e produtiva as sessões de terapia. Essa aplicação visa oferecer uma maior gama de atividades para trabalhar com seus pacientes e, ao paciente, o aplicativo irá fazer com que ele aprenda, brinque e ainda tenha os estímulos ou bonificações ao acertar ou completar uma tarefa de forma correta.

Este trabalho tem como objetivo a criação de um projeto de software para construção de uma ferramenta mobile de auxílio ao desenvolvimento de habilidades comportamentais em crianças com TEA, fazendo uso de técnicas de e-learning e gamificação. As metas deste trabalho são:

\begin{itemize}
	\item fazer o levantamento de mecânicas de gamificação;
	
	\item relacionar cada mecânica de gamificação para o público específico; e
	
	\item definir e produzir os artefatos de software necessários para a construção da ferramenta.
\end{itemize}

Este trabalho de estágio estará dividida da seguinte forma:
 	\begin{itemize}
		\item Capítulo 2, Referencial Teórico. Apresenta a base conceitual que envolve o escopo deste trabalho. São descritos os seguintes assuntos: autismo, gamificação e E-learning.
		
		\item Capítulo 3, Metodologia. Apresenta a metodologia aplicada no desenvolvimento do presente trabalho.
		
		\item Capítulo 4, Trabalhos Relacionados. São apresentados artigos e aplicativos que abordam o tema de estudo desse trabalho.
		
		\item Capítulo 4, Resultados e Discussão. São apresentados os resultados obtidos durante o estudo e elaboração deste trabalho.
		
		\item Capítulo 5, Conclusão. Apresenta as contribuições deste trabalho e indica sugestões para trabalhos futuros.
	\end{itemize}

