% ---
% Capitulo de METODOLOGIA
% ---
\chapter{Metodologia}\label{cap:metodologia}

Este trabalho trata-se de um estudo descritivo e exploratório sobre realidade aumentada e geolocalização por meio de dispositivos móveis como os \textit{smartphones}. Para isso, a obtenção do conhecimento sobre estes assuntos foram dadas por materiais bibliográfico, o qual foi adquirido através de livros e artigos científicos, assim como, por meio das documentações fornecida pelas próprias mantenedoras das ferramentas e aplicativos citado.

O objetivo deste trabalho é tomar conhecimento a respeito das bibliotecas que ajudam a desenvolver aplicações usando tecnologias de realidade aumentada, foram selecionadas quatro, sendo elas: ARToolKit\footnote{Acesse: http://www.hitl.washington.edu/artoolkit/}, Layar\footnote{Acesse:https://www.layar.com/}, Vuforia\footnote{Acesse: https://www.vuforia.com/} e ARCore\footnote{Acesse: https://developers.google.com/ar/}, e com isso foram extraídos características para a definição de qual será usada no desenvolvimento da aplicação.

Será definido a biblioteca de RA acordo com critérios que estão disponíveis na tabela \ref{tabComparativa}, onde são eles:

\begin{itemize}
	\item O custo que o desenvolvedor terá para usar a biblioteca;
	\item A plataforma ou IDE que pode ser usada para o desenvolvimento;
	\item As plataformas de execução, onde é levado em conta principalmente se a mesma possui suporte para mobile;
	\item A Documentação que aquela biblioteca possui, o que facilitará na hora de o desenvolvedor conhecer os comandos necessários.
	\item A linguagem que pode ser usada no desenvolvimento.
\end{itemize}
