% ----------------------------------------------------------
% Introdução (exemplo de capítulo sem numeração, mas presente no Sumário)
% ----------------------------------------------------------

\chapter{Introdução}\label{cap:introducao}

%-------------------------------------------------------------------------------------
%	Mundo mobile - OK
%	evolução dos dispositivos - OK
%	Benefícios para a população - OK
%	Surgimentos de novas tecnologias como as RA's - OK
%	
%	Proposta de estudo de frameworks e bibliotecas
%-------------------------------------------------------------------------------------
O uso de \textit{smartphone} vem crescendo de forma alarmante. Segundo o \citeonline{estadao:2018}, um estudo realizado pela fundação Getúlio Vargas aponta que no Brasil superou a marca de um \textit{smartphone} por pessoa. Com um total de 220 milhões de celulares, a pesquisa indica ainda que no ano de 2017 foram vendidos um total de 48 milhões de aparelhos. Segundo a \citeonline{gsma:2017}, no mundo 5 bilhões de pessoas possui um ou mais \textit{smartphone}, a pesquisa mostra ainda a china como líder do \textit{ranking} com mais de 1 bilhão de aparelhos ativos.

A população mundial vem aceitando e usando consideravelmente esses tipos de aparelhos, pois eles agregam funcionalidades que ajudam e beneficiam o usuário em suas tarefas diárias, substituindo assim outros tipos de equipamentos e fazendo com que não se precise carregar diferentes tipos de equipamentos. Com o celular inteligente é possível se localizar usando mapas, ter acesso a internet, enviar e receber e-mails, jogar, usar como despertador, tirar fotos e gravar vídeos dentre outras diversas funcionalidades. Vale ainda ressaltar que o envio de mensagens se tonou ainda mais fácil, com aplicativos de mensagem instantânea como \textit{Whatsapp}, Telegram e Facebook Messages.

Nos últimos anos os aparelhos de \textit{smartphone} tem ganhados cada vez mais funcionalidades e diversas novas tecnologias tem sido atrelada a esses dispositivos, como por exemplo a tecnologia de realidade aumentada que segundo definição de  \citeonline{kirner:2007}, se trata do enriquecimento de um determinado ambiente real com o uso de objetos virtuais e com seu funcionamento dado em tempo real, outra tecnologia bastante popular é a geolocalização, na qual todo \textit{smartphone} já sai de fábrica com aplicações e suporte a mapas e localização em tempo real, Outro recurso que tem ganhado o gosto dos usuários é a realidade virtual, que segundo \citeonline{tori:2006}, se trata de uma interface avançada onde o usuário pode acessar aplicações dentro do computador, onde se tem um ambiente tridimensional, e o usuário por sua vez pode interagir com esses.
 
\section{Objetivos}
\subsection{Objetivo Geral }
	Realizar um estudo comparativo entre as bibliotecas existentes na área de realidade aumentada e geolocalização.

\subsection{Objetivos Específicos}
\begin{itemize}
	\item Identificar API's de RA  e Geolocalização disponíveis atualmente;
	\item realizar um comparativo entre as bibliotecas existentes;
	\item estudar a documentação das API's analisando a compatibilidade de uso e custo;
	\item desenvolver um estudo de caso envolvendo as bibliotecas e tecnologias estudadas.
\end{itemize}
