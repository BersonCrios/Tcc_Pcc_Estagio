\chapter{Conclusão}\label{cap:conclusao}
Este trabalho teve como principal objetivo o estudo sobre bibliotecas de realidade aumentada para assim ter-se o conhecimento sobre qual será usada para o desenvolvimento de uma biblioteca que faz o uso de realidade aumentada e geolocalização. Com os dados recolhidos sobre as bibliotecas estudadas,foi possível perceber que a biblioteca lançada pelo Google no inicio do ano de 2018, chamada ARCore, que é uma biblioteca fornecida gratuitamente, como algumas vantagens tem-se a grande gama de ambientes para desenvolvimento, como unity, android studio e unreal, sendo ainda possível com ela o desenvolvimento de aplicações de realidade aumentada para sistemas operacionais móveis da apple.

A biblioteca de RA foi escolhida de acordo com a tabela \ref{tabComparativa}, onde segundo alguns critérios que foram estudados no referencial teórico, ficou decidido assim o uso da biblioteca ARCore, que tem atributos mais atraentes para estudo e comercialmente, o que possibilita a criação de apps com suporte a RA de maneira mais simples e sem o uso de marcadores.

Como se trata de uma biblioteca lançada recentemente ainda não teve sua liberação para o grande número de celulares, assim sendo ainda existem poucos modelos que suportam essa biblioteca, sendo preciso também ter a a ultima versão da biblioteca de openGL e do sistema operacional android. 